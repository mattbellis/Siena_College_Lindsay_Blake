\documentclass[prb,preprint]{revtex4-1} 
% The line above defines the type of LaTeX document.
% Note that AJP uses the same style as Phys. Rev. B (prb).

% The % character begins a comment, which continues to the end of the line.

\usepackage{amsmath}  % needed for \tfrac, \bmatrix, etc.
\usepackage{amsfonts} % needed for bold Greek, Fraktur, and blackboard bold
\usepackage{graphicx} % needed for figures

\begin{document}

% Be sure to use the \title, \author, \affiliation, and \abstract macros
% to format your title page.  Don't use lower-level macros to  manually
% adjust the fonts and centering.

\title{Tests of a nearest-neighbor fitting algorithm using MINUIT}

\author{Matt Bellis}
\email{mbellis@siena.edu} % optional
\author{Lindsay Blake}
\email{lm13blak@siena.edu} % optional
%\altaffiliation[]{101 Main Street, Anytown, USA} % optional second address
% If there were a second author at the same address, we would put another 
% \author{} statement here.  Don't combine multiple authors in a single
% \author statement.
\affiliation{Department of Physics And Astronomy, Siena College, Loudonville, NY 12211}
% Please provide a full mailing address here.

%\author{David P. Jackson}
%\email{ajp@dickinson.edu}
%\affiliation{Department of Physics, Dickinson College, Carlisle, PA 17013}

% See the REVTeX documentation for more examples of author and affiliation lists.

\date{\today}

\begin{abstract}
Fitting data to some model is a standard part of most physics analysis 
and usually involves a $\chi^2$ technique or maximum likelihood method. 
In both these cases, an analytic solution for the model is usually 
readily available and so a best-fit line or probability density function (PDF)
can be calculated. However, sometimes an analytic solution is not available,
but instead datasets based on the model can be generated and used
as templates for fitting, usually as binned histograms. This approach
becomes more challenging for multidimensional datasets. We present a
different approaching using the density of nearest neighbors as a 
replacement for a standard PDF, but still using the standard minimization
machinery, MINUIT. We find...
\end{abstract}
% AJP requires an abstract for all regular article submissions.

\maketitle % title page is now complete


%%%%%%%%%%%%%%%%%%%%%%%%%%%%%%%%%%%%%%%%%%%%%%%%%%%%%%%%%%%%%%%%%%%%%%%%%%%%%%%%
%%%%%%%%%%%%%%%%%%%%%%%%%%%%%%%%%%%%%%%%%%%%%%%%%%%%%%%%%%%%%%%%%%%%%%%%%%%%%%%%

\section{Introduction \label{sec:intro}} 
%%%%%%%%%%%%%%%%%%%%%%%%%%%%%%%%%%%%%%%%%%%%%%%%%%%%%%%%%%%%%%%%%%%%%%%%%%%%%%%%
% Introduction
%%%%%%%%%%%%%%%%%%%%%%%%%%%%%%%%%%%%%%%%%%%%%%%%%%%%%%%%%%%%%%%%%%%%%%%%%%%%%%%%



\section{Conclusions \label{sec:conclusions}} 
%%%%%%%%%%%%%%%%%%%%%%%%%%%%%%%%%%%%%%%%%%%%%%%%%%%%%%%%%%%%%%%%%%%%%%%%%%%%%%%%
% Conclusions
%%%%%%%%%%%%%%%%%%%%%%%%%%%%%%%%%%%%%%%%%%%%%%%%%%%%%%%%%%%%%%%%%%%%%%%%%%%%%%%%



\begin{acknowledgments}

We gratefully acknowledge NSF support XXX.....

\end{acknowledgments}


\begin{thebibliography}{99}
% The numeral (here 99) in curly braces is nominally the number of entries in
% the bibliography. It's supposed to affect the amount of space around the
% numerical labels, so only the number of digits should matter--and even that
% seems to make no discernible difference.

%\bibitem{latexsite} \LaTeX\ Project Web Site, \url{<http://www.latex-project.org/>}.
%
%\bibitem{wikibook} \textit{\LaTeX} (Wikibook), \url{<http://en.wikibooks.org/wiki/LaTeX/>}.
%
%\bibitem{latexbook}Helmut Kopka and Patrick W. Daly, \textit{A Guide to
%\LaTeX}, 4th edition (Addison-Wesley, Boston, 2004).
%
%\bibitem{revtex} REV\TeX\ 4 Home Page, \url{<https://authors.aps.org/revtex4/>}.
%
%\bibitem{cloudLaTeX} On the other hand, you can avoid the installation process
%entirely by using a cloud-based \LaTeX\ processor such as ShareLaTeX,
%\url{<https://www.sharelatex.com/>}, or write\LaTeX, \url{<https://www.writelatex.com/>}.
%
%\bibitem{nevermindlogic} In typography, aesthetics often takes precedence over logic.
%
%\bibitem{FontEncodingComment} Please don't try to handle foreign characters 
%and accents with the \texttt{inputenc} and \texttt{fontenc} packages, which 
%are incompatible with AJP's editing process.
%
%\bibitem{wikimathpage} See the Mathematics chapter of Ref.~\onlinecite{wikibook}
%for an excellent overview of math symbols and equations, with examples.
%
%\bibitem{labelnames} Thinking up a good label name takes a moment, but 
%it's worth the trouble; we strongly advise against using labels like 
%\texttt{eq2}, which become extremely confusing after you decide to add 
%another equation before Eq.~(\ref{deriv}).
%
%\bibitem{footnotes} You need to process a file twice to get the counters correct.
%
%\bibitem{mermin} N. David Mermin, ``What's wrong with these equations?,'' 
%Phys. Today \textbf{42} (10), 9--11 (1989).  
%% Note that the issue number (10) in this citation is required, because
%% each issue of Physics Today starts over with page 1.  Also note the use of
%% an en-dash (--), not a hyphen (-), for the page range.
%
%\bibitem{editorsite} American Journal of Physics Editor's Web Site, 
%\url{<http://ajp.dickinson.edu>}.
%
%\bibitem{feynman} Richard P. Feynman, Robert B. Leighton, and Matthew Sands, 
%\textit{The Feynman Lectures on Physics, Vol.\ 1} (Addison-Wesley, 1964), p.~3-10.
%% Note that this book is paginated by chapter; "3-10" is a single page reference
%% that uses a hyphen, not a range of pages that would us an en-dash (--).
%
%\bibitem{noBIBTeX} Many \LaTeX\ users manage their bibliographic data with 
%a tool called BIB\TeX.  Unfortunately, AJP cannot accept BIB\TeX\ files; all 
%bibliographic references must be incorporated into the manuscript file
%as shown here, at least when you send an editable file for production.
%
%\bibitem{dyson} Freeman J. Dyson, ``Feynman's proof of the Maxwell equations,''
%Am. J. Phys. \textbf{58} (3), 209--211.  
%% The issue number (3) in this citation is optional, because AJP's pagination 
%% is by volume.
%
%\bibitem{examplevolume} M. R. Flannery, ``Elastic scattering,'' in 
%\textit{Atomic, Molecular, and Optical Physics Handbook}, edited by
%G. W. F. Drake (AIP Press, New York, 1996), p.~520.
%%
%\bibitem{AIPstylemanual} \textit{AIP Style Manual}, 4th edition (American 
%Institute of Physics, New York, 1990). Available online at 
%\url{<http://www.aip.org/pubservs/style/4thed/toc.html>}. Although parts of 
%it have been made out of date by advancing technology, most of this manual 
%is still as useful as ever. Just be sure to follow AJP's specific rules
%whenever they conflict with those in the manual.

\end{thebibliography}

% If your manuscript is conditionally accepted, the editors will ask you to
% submit your editable LaTeX source file.  Before doing so, you should move
% all tables and figure captions to the end, as shown below.  Tables come 
% first, followed by figure captions (with figure inclusions commented-out).
% Figures should be submitted as separate files, collected with the
% LaTeX file into a single .zip archive.

%\newpage   % Start a new page for tables

%\begin{table}[h!]
%\centering
%\caption{Elementary bosons}
%\begin{ruledtabular}
%\begin{tabular}{l c c c c p{5cm}}
%Name & Symbol & Mass (GeV/$c^2$) & Spin & Discovered & Interacts with \\
%\hline
%Photon & $\gamma$ & \ \ 0 & 1 & 1905 & Electrically charged particles \\
%Gluons & $g$ & \ \ 0 & 1 & 1978 & Strongly interacting particles (quarks and gluons) \\
%Weak charged bosons & $W^\pm$ & \ 82 & 1 & 1983 & Quarks, leptons, $W^\pm$, $Z^0$, $\gamma$ \\
%Weak neutral boson & $Z^0$ & \ 91 & 1 & 1983 & Quarks, leptons, $W^\pm$, $Z^0$ \\
%Higgs boson & $H$ & 126 & 0 & 2012 & Massive particles (according to theory) \\
%\end{tabular}
%\end{ruledtabular}
%\label{bosons}
%\end{table}

%\newpage   % Start a new page for figure captions

%\section*{Figure captions}

%\begin{figure}[h!]
%\centering
%\includegraphics{GasBulbData.eps}   % This line stays commented-out
%\caption{Pressure as a function of temperature for a fixed volume of air.  
%The three data sets are for three different amounts of air in the container. 
%For an ideal gas, the pressure would go to zero at $-273^\circ$C.  (Notice
%that this is a vector graphic, so it can be viewed at any scale without
%seeing pixels.)}

%\label{gasbulbdata}
%\end{figure}

%\begin{figure}[h!]
%\centering
%\includegraphics[width=5in]{ThreeSunsets.jpg}   % This line stays commented-out
%\caption{Three overlaid sequences of photos of the setting sun, taken
%near the December solstice (left), September equinox (center), and
%June solstice (right), all from the same location at 41$^\circ$ north
%latitude. The time interval between images in each sequence is approximately
%four minutes.}
%\label{sunsets}
%\end{figure}

\end{document}
